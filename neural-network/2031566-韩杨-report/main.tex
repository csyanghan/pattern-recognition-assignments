%%%%%%%%%%%%%%%%%%%%%%%%%%%%%%%%%%%%%%%%%
% Lachaise Assignment
% LaTeX Template
% Version 1.0 (26/6/2018)
%
% This template originates from:
% http://www.LaTeXTemplates.com
%
% Authors:
% Marion Lachaise & François Févotte
% Vel (vel@LaTeXTemplates.com)
%
% License:
% CC BY-NC-SA 3.0 (http://creativecommons.org/licenses/by-nc-sa/3.0/)
% 
%%%%%%%%%%%%%%%%%%%%%%%%%%%%%%%%%%%%%%%%%

%----------------------------------------------------------------------------------------
%	PACKAGES AND OTHER DOCUMENT CONFIGURATIONS
%----------------------------------------------------------------------------------------

\documentclass{article}

%%%%%%%%%%%%%%%%%%%%%%%%%%%%%%%%%%%%%%%%%
% Lachaise Assignment
% Structure Specification File
% Version 1.0 (26/6/2018)
%
% This template originates from:
% http://www.LaTeXTemplates.com
%
% Authors:
% Marion Lachaise & François Févotte
% Vel (vel@LaTeXTemplates.com)
%
% License:
% CC BY-NC-SA 3.0 (http://creativecommons.org/licenses/by-nc-sa/3.0/)
% 
%%%%%%%%%%%%%%%%%%%%%%%%%%%%%%%%%%%%%%%%%

%----------------------------------------------------------------------------------------
%	PACKAGES AND OTHER DOCUMENT CONFIGURATIONS
%----------------------------------------------------------------------------------------

\usepackage{amsmath,amsfonts,stmaryrd,amssymb} % Math packages

\usepackage{enumerate} % Custom item numbers for enumerations

\usepackage[ruled]{algorithm2e} % Algorithms

\usepackage[framemethod=tikz]{mdframed} % Allows defining custom boxed/framed environments

\usepackage{listings} % File listings, with syntax highlighting
\lstset{
	basicstyle=\ttfamily, % Typeset listings in monospace font
}

%----------------------------------------------------------------------------------------
%	DOCUMENT MARGINS
%----------------------------------------------------------------------------------------

\usepackage{geometry} % Required for adjusting page dimensions and margins

\geometry{
	paper=a4paper, % Paper size, change to letterpaper for US letter size
	top=2.5cm, % Top margin
	bottom=3cm, % Bottom margin
	left=2.5cm, % Left margin
	right=2.5cm, % Right margin
	headheight=14pt, % Header height
	footskip=1.5cm, % Space from the bottom margin to the baseline of the footer
	headsep=1.2cm, % Space from the top margin to the baseline of the header
	%showframe, % Uncomment to show how the type block is set on the page
}

%----------------------------------------------------------------------------------------
%	FONTS
%----------------------------------------------------------------------------------------

\usepackage[utf8]{inputenc} % Required for inputting international characters
\usepackage[T1]{fontenc} % Output font encoding for international characters

\usepackage{XCharter} % Use the XCharter fonts

%----------------------------------------------------------------------------------------
%	COMMAND LINE ENVIRONMENT
%----------------------------------------------------------------------------------------

% Usage:
% \begin{commandline}
%	\begin{verbatim}
%		$ ls
%		
%		Applications	Desktop	...
%	\end{verbatim}
% \end{commandline}

\mdfdefinestyle{commandline}{
	leftmargin=10pt,
	rightmargin=10pt,
	innerleftmargin=15pt,
	middlelinecolor=black!50!white,
	middlelinewidth=2pt,
	frametitlerule=false,
	backgroundcolor=black!5!white,
	frametitle={Command Line},
	frametitlefont={\normalfont\sffamily\color{white}\hspace{-1em}},
	frametitlebackgroundcolor=black!50!white,
	nobreak,
}

% Define a custom environment for command-line snapshots
\newenvironment{commandline}{
	\medskip
	\begin{mdframed}[style=commandline]
}{
	\end{mdframed}
	\medskip
}

%----------------------------------------------------------------------------------------
%	FILE CONTENTS ENVIRONMENT
%----------------------------------------------------------------------------------------

% Usage:
% \begin{file}[optional filename, defaults to "File"]
%	File contents, for example, with a listings environment
% \end{file}

\mdfdefinestyle{file}{
	innertopmargin=1.6\baselineskip,
	innerbottommargin=0.8\baselineskip,
	topline=false, bottomline=false,
	leftline=false, rightline=false,
	leftmargin=2cm,
	rightmargin=2cm,
	singleextra={%
		\draw[fill=black!10!white](P)++(0,-1.2em)rectangle(P-|O);
		\node[anchor=north west]
		at(P-|O){\ttfamily\mdfilename};
		%
		\def\l{3em}
		\draw(O-|P)++(-\l,0)--++(\l,\l)--(P)--(P-|O)--(O)--cycle;
		\draw(O-|P)++(-\l,0)--++(0,\l)--++(\l,0);
	},
	nobreak,
}

% Define a custom environment for file contents
\newenvironment{file}[1][File]{ % Set the default filename to "File"
	\medskip
	\newcommand{\mdfilename}{#1}
	\begin{mdframed}[style=file]
}{
	\end{mdframed}
	\medskip
}

%----------------------------------------------------------------------------------------
%	NUMBERED QUESTIONS ENVIRONMENT
%----------------------------------------------------------------------------------------

% Usage:
% \begin{question}[optional title]
%	Question contents
% \end{question}

\mdfdefinestyle{question}{
	innertopmargin=1.2\baselineskip,
	innerbottommargin=0.8\baselineskip,
	roundcorner=5pt,
	nobreak,
	singleextra={%
		\draw(P-|O)node[xshift=1em,anchor=west,fill=white,draw,rounded corners=5pt]{%
		Question \theQuestion\questionTitle};
	},
}

\newcounter{Question} % Stores the current question number that gets iterated with each new question

% Define a custom environment for numbered questions
\newenvironment{question}[1][\unskip]{
	\bigskip
	\stepcounter{Question}
	\newcommand{\questionTitle}{~#1}
	\begin{mdframed}[style=question]
}{
	\end{mdframed}
	\medskip
}

%----------------------------------------------------------------------------------------
%	WARNING TEXT ENVIRONMENT
%----------------------------------------------------------------------------------------

% Usage:
% \begin{warn}[optional title, defaults to "Warning:"]
%	Contents
% \end{warn}

\mdfdefinestyle{warning}{
	topline=false, bottomline=false,
	leftline=false, rightline=false,
	nobreak,
	singleextra={%
		\draw(P-|O)++(-0.5em,0)node(tmp1){};
		\draw(P-|O)++(0.5em,0)node(tmp2){};
		\fill[black,rotate around={45:(P-|O)}](tmp1)rectangle(tmp2);
		\node at(P-|O){\color{white}\scriptsize\bf !};
		\draw[very thick](P-|O)++(0,-1em)--(O);%--(O-|P);
	}
}

% Define a custom environment for warning text
\newenvironment{warn}[1][Warning:]{ % Set the default warning to "Warning:"
	\medskip
	\begin{mdframed}[style=warning]
		\noindent{\textbf{#1}}
}{
	\end{mdframed}
}

%----------------------------------------------------------------------------------------
%	INFORMATION ENVIRONMENT
%----------------------------------------------------------------------------------------

% Usage:
% \begin{info}[optional title, defaults to "Info:"]
% 	contents
% 	\end{info}

\mdfdefinestyle{info}{%
	topline=false, bottomline=false,
	leftline=false, rightline=false,
	nobreak,
	singleextra={%
		\fill[black](P-|O)circle[radius=0.4em];
		\node at(P-|O){\color{white}\scriptsize\bf i};
		\draw[very thick](P-|O)++(0,-0.8em)--(O);%--(O-|P);
	}
}

% Define a custom environment for information
\newenvironment{info}[1][Info:]{ % Set the default title to "Info:"
	\medskip
	\begin{mdframed}[style=info]
		\noindent{\textbf{#1}}
}{
	\end{mdframed}
}
 % Include the file specifying the document structure and custom commands
\setlength{\parindent}{0pt}
%----------------------------------------------------------------------------------------
%	ASSIGNMENT INFORMATION
%----------------------------------------------------------------------------------------

\title{PATTERN RECOGNITION: Assignment \#2} % Title of the assignment

\author{2031566 Yang Han\\ \texttt{hanyang\_sh@tongji.edu.cn}} % Author name and email address

\date{School of Software Engineering, Tongji University  --- \today} % University, school and/or department name(s) and a date

%----------------------------------------------------------------------------------------

\begin{document}

\maketitle % Print the title

%----------------------------------------------------------------------------------------
%	INTRODUCTION
%----------------------------------------------------------------------------------------

\section*{Introduction} % Unnumbered section

In this assignment, I will implement neural network and evaluate it on Seeds Data Set $\footnote{http://archive.ics.uci.edu/ml/datasets/seeds}$ and Iris Data Set $\footnote{http://archive.ics.uci.edu/ml/datasets/Iris}$.


\section{Dataset Introduction} % Numbered section

\subsection{Seeds Dataset}
The dataset belongs to three different varieties of wheat and is consisted of seven geometric parameters, 1. area A,
2. perimeter P,
3. compactness C
4. length of kernel,
5. width of kernel,
6. asymmetry coefficient
7. length of kernel groove. All of these parameters were real-valued continuous.

\subsection{Iris Dataset}
Iris Dataset is perhaps the best known database to be found in the pattern recognition literature, 
The data set contains 3 classes of 50 instances each, where each class refers to a type of iris plant. One class is linearly separable from the other 2; the latter are NOT linearly separable from each other.


\section{Data Preprocess}
As I show in the $preprocess.ipynb$ file, I mainly do three things:
\begin{enumerate}[1.]
  \item Read the raw file content, cast the str type to float type save save them to csv format.
  \item Map categories to ordered numbers(eg. 0,1,2...).
  \item In order to eliminate the dimensional influence between indicators, data standardization
   is required to solve the comparability between data indicators.
\end{enumerate}

\section{Program Modules}

\subsection{Theory}
The main part of error backpropagate algorithm is how to update the weight, as the silde shows that:
$$\Delta w_{ij} = \eta g_i b_h$$
$$\Delta \theta_{j} = - \eta g_i$$
$$\Delta v_{ij} = \eta e_h x_i$$
$$\Delta \gamma_{h} = - \eta e_h$$
And $g_j$ and $e_h$ can be denoted by the following formula:
$$g_j = \tilde{y}_j^k(1-\tilde{y}_j^k)(y_j^k-\tilde{y}_j^k)$$
$$e_h = b_h(1-b_h)\sum_{j=1}^{l} {w_{hj} g_j}$$

\subsection{Code}
In the part, i design a class named \textbf{NeuralNetwork} which accepted four parameters:$n\_inputs$ denotes the number of
input, $n\_hiddens$denotes the number of neurons in hidden layer, $n\_outputs$denotes the number of output, 
$lr$denotes the learning rate and have a default value 0.1.

This class mainly has a attribute named \textbf{network} which store the weights, bias, every neuron's output and the value of $g_i$ and $b_h$ mentiond above.

\textbf{I implement the interface in Pytorch way}, it also mainly has four methods. 1.\textbf{forward} is to calculate the output of a neuron.
2.\textbf{error\_backpropagate} is the main error backpropagate algorithm, 3.\textbf{step} update the weight and bias stored in the \textbf{network}
attribute. 4.\textbf{criterion} calculate the loss.

\section{Results}
I use \textbf{kFold Cross Validation} to get a more normal result.I use 5\_fold in this experiment, In the seeds dataset($n_inputs=7, n_hiddens=5, n_outputs=3$) with epochs=100, 
I get a result:
\begin{commandline}
  \begin{verbatim}
seeds dataset, k_fold: 1, accurancy: 88.57142857142857%
seeds dataset, k_fold: 2, accurancy: 86.11111111111111%
seeds dataset, k_fold: 3, accurancy: 86.48648648648648%
seeds dataset, k_fold: 4, accurancy: 90.9090909090909%
seeds dataset, k_fold: 5, accurancy: 97.05882352941177%
	\end{verbatim}
\end{commandline}
In the iris dataset($n\_inputs=4, n\_hiddens=5, n\_outputs=3$) with epochs=100, I get a result:
\begin{commandline}
  \begin{verbatim}
iris dataset, k_fold: 1, accurancy: 100.0%
iris dataset, k_fold: 2, accurancy: 100.0%
iris dataset, k_fold: 3, accurancy: 100.0%
iris dataset, k_fold: 4, accurancy: 100.0%
iris dataset, k_fold: 5, accurancy: 100.0%
	\end{verbatim}
\end{commandline}
\section{Limitations and Inprovements}

The main limitation is the number of layers of neural network is 2, I think a neural network with deeper depth will
have better perfomance, but it's difficult to calculate derivation manually, we can use compution graph with Pytorch 
or Tensorflow to get the derivation quickly and update the weights. And the width of the hidden layer is also a limitation,
neural network with wider width will also have better accurancy, but the seeds dataset and iris dataset is
simple, and we can get good results approximately at 90\% and 100\%.
%----------------------------------------------------------------------------------------

\end{document}
