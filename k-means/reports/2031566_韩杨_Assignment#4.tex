\documentclass{article}

\usepackage{fancyhdr}
\usepackage{extramarks}
\usepackage{amsmath}
\usepackage{amssymb}
\usepackage{amsthm}
\usepackage{amsfonts}
\usepackage{tikz}
\usepackage[plain]{algorithm}
\usepackage{algpseudocode}
\usepackage{enumerate}
\usepackage{color}
\usetikzlibrary{automata,positioning}

%
% Basic Document Settings
%

\topmargin=-0.45in
\evensidemargin=0in
\oddsidemargin=0in
\textwidth=6.5in
\textheight=9.0in
\headsep=0.25in

\linespread{1.1}

\pagestyle{fancy}
\lhead{\hmwkAuthorName}
\chead{\hmwkClass}
\rhead{\hmwkTitle}
\lfoot{\Address}
\rfoot{Page \thepage}
\cfoot{}

\renewcommand\headrulewidth{0.4pt}
\renewcommand\footrulewidth{0.4pt}

\setlength\parindent{0pt}

%
% Create Problem Sections
%

\newcommand{\enterProblemHeader}[1]{
    \nobreak\extramarks{}{Problem \arabic{#1} continued on next page\ldots}\nobreak{}
    \nobreak\extramarks{Problem \arabic{#1} (continued)}{Problem \arabic{#1} continued on next page\ldots}\nobreak{}
}

\newcommand{\exitProblemHeader}[1]{
    \nobreak\extramarks{Problem \arabic{#1} (continued)}{Problem \arabic{#1} continued on next page\ldots}\nobreak{}
    \stepcounter{#1}
    \nobreak\extramarks{Problem \arabic{#1}}{}\nobreak{}
}

\setcounter{secnumdepth}{0}
\newcounter{partCounter}
\newcounter{homeworkProblemCounter}
\setcounter{homeworkProblemCounter}{1}
\nobreak\extramarks{Problem \arabic{homeworkProblemCounter}}{}\nobreak{}

\newenvironment{homeworkProblem}{
    \section{Problem \arabic{homeworkProblemCounter}}
    \setcounter{partCounter}{1}
    \enterProblemHeader{homeworkProblemCounter}
}{
    \exitProblemHeader{homeworkProblemCounter}
}

%
% Homework Details
%   - Title
%   - Due date
%   - Class
%   - Section/Time
%   - Instructor
%   - Author
%

\newcommand{\hmwkTitle}{Assignment\#4}
\newcommand{\hmwkDueDate}{\today}
\newcommand{\hmwkClass}{Pattern Recognition}
\newcommand{\hmwkClassTime}{}
\newcommand{\Address}{SSE TongJi University}
\newcommand{\hmwkClassInstructor}{Professor Ying Shen}
\newcommand{\hmwkAuthorName}{2031566/Yang Han}

%
% Title Page
%

\title{
    \vspace{2in}
    \textmd{\textbf{\hmwkClass\\ \hmwkTitle}}\\
    \normalsize\vspace{0.1in}\small{By\ \textit{\hmwkClassInstructor\ \hmwkClassTime }}\\
 %   \vspace{0.1in}\small{\Address \\ Due\ on\ \hmwkDueDate\  }
    \vspace{3in}
}

\author{\textbf{\hmwkAuthorName}}
\date{\today}

\renewcommand{\part}[1]{\textbf{\large Part \Alph{partCounter}}\stepcounter{partCounter}\\}

%
% Various Helper Commands
%

% Useful for algorithms
\newcommand{\alg}[1]{\textsc{\bfseries \footnotesize #1}}

% For derivatives
\newcommand{\deriv}[1]{\frac{\mathrm{d}}{\mathrm{d}x} (#1)}

% For partial derivatives
\newcommand{\pderiv}[2]{\frac{\partial}{\partial #1} (#2)}

% Integral dx
\newcommand{\dx}{\mathrm{d}x}

% Alias for the Solution section header
\newcommand{\solution}{\textbf{\large Solution}}

% Probability commands: Expectation, Variance, Covariance, Bias
\newcommand{\E}{\mathrm{E}}
\newcommand{\Var}{\mathrm{Var}}
\newcommand{\Cov}{\mathrm{Cov}}
\newcommand{\Bias}{\mathrm{Bias}}

\begin{document}
	\maketitle
	\newpage
	\begin{enumerate}
		\item
		Kernel computation cost
		
		1).Consider we have a two-dimensional input space such that the input vector is $x = (x_1, x_2)^T$
		Define the feature mapping $\phi(x) = (x_1^2, \sqrt{2}x_1x_2, x_2^2)^T$, What is the corresponding kernel function,
		i.e. $K(x,z)$? Do not level $\phi(x)$ in your final answer.

		\solution:
		
		$K(x,z) = \phi(x)^T \cdot \phi(z) = (x_1^2, \sqrt{2}x_1x_2, x_2^2) \cdot (z_1^2, \sqrt{2}z_1z_2, z_2^2)^T = x_1^2z_1^2+ 2x_1x_2z_1z_2+x_2^2z_2^2$


		2).Suppose we want to compute the value of the kernel function$K(x,z)$ from the previous
		question, on two vectors $x,z \in \mathbb{R}^2$.How many additions and multiplications are needed if you 
		
		(a) map the input vector to the feature space and the perform the dot product on the mapped features?

		\solution:
		
		To compute this projection, we need 4 multiplications, there exists two point, so need 8 multiplications,
		and then the dot product itself requires 3 multiplications and 2 additions.so we totally need
		11 multiplications and 2 additions.

		(b) compute through the kernel function you derived in question 1?

		\solution:

		$K(x,z) = x_1^2z_1^2+ 2x_1x_2z_1z_2+x_2^2z_2^2 = (x_1z_1 + x_2z_2)^2$ \\
		From the above equation, we can easily find that we need 3 multiplications and 1 additions.

		\item
		Kernel functions, Consider the following kernel function: \\
		$
			K(x,x^{'})=
				\begin{cases}
				1 &\mbox{if $x=x^{'}$}\\\
				0 &\mbox{otherwise}
				\end{cases}
		$

		1).Prove this is a legal kernel. That is, describe an implicit mapping: $\Phi:X \rightarrow \mathbb{R}^m$ such that
		$ K(x, x^{'}) = \Phi(x) * \Phi(x^{'})$. (You may assume the instance space $X$ is finite.)
		
		\solution:

		We get get the kernel matrix 
		$ K = 
				\begin{pmatrix}
					1 & 0 & \cdots & 0 \\
					0 & 1 & \cdots & 0 \\
					\vdots & \vdots & \ddots & \vdots \\
					0 & 0 & \cdots & 1 
				\end{pmatrix}
		$, and $k$ is a is positive semidefinite matrix, so this is a legal kernel.
		
		2).In this kernel space, any labeling of points in $X$ will be linearly separable. Justify this claim.

		\solution:

		$K(x_i,x_j)=\Phi(x_i)^T \cdot \Phi(x_j) = \begin{cases}
			1 &\mbox{if $x_i=x_j$}\\\
			0 &\mbox{otherwise}
			\end{cases}
		$
		Then $\Phi(x_i) = e_i$, $e_i$ is the column vertor which the $i$th element is 1 as $[0,0,0,...,1..., 0]^T$

		So the decision boundary is $f(x) = W^T \Phi(x) = W^T e_i = W_i = y_i$, all data point will be  linearly separable.
		
		3).Since all labelings are linearly separable, this kernel seems perfect for learning any target
		function. Why is this actually a bad idea?
		
		\solution:

		Because $W$ will overfit the data.
		
	\end{enumerate}

\end{document}